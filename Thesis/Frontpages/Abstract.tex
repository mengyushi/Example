% The Abstract Page
\clearpage
\addtotoc{Abstract}  % Add the "Abstract" page entry to the Contents
\abstract{
    \addtocontents{toc}{\vspace{1em}}  % Add a gap in the Contents,
                                        %for aesthetics

    %The Thesis Abstract is written here (and usually kept to just this page).
    %The page is kept centered vertically so can expand into the blank space above the title too \ldots

    Nowadays, many concrete structures are exposed to hazardous expansion, which results in a less structural capacity. Meanwhile, as the production process advanced, expansion does occur in those made from some newly developed methods, such as high temperature cured precast concrete elements. However, the mechanism and degree of losses strength under various kinds of expansion are still not very clear. Based on the experimental works, it is not easy to understand the behavior due to the delicate influence factors, difficulties in inner inspection and long term required by the expansion to develop.
    
    This study aims to develop a three-dimensional numerical simulation system of concrete expansion, caused by ASR and DEF, for searching the trend in their cracking patterns, and strength capacity remained.

    In this study, the expansion behavior of concrete is simulated base on Rigid Body Spring Model(RBSM). To replicate the process of expansion, the initial strain is introduced between elements, then the stress propagates from there, leading to changes in the whole concrete structure. Base on the previous studies, the ASR and DEF expansion behavior is able to be simulated by the RBSM method, though the details are still required to be improved. A new method of strain distribution has been introduced into DEF and ASR simulation to improve the similarities in structural behavior with experimental results.

    Also, the changing in mechanical properties are discussed, which have not to be done before. After the expansion behavior is well simulated, its loss in mechanical properties is also able to be simulated. For both ASR and DEF expansion, although countermeasures have yielded considerable success in minimize the risk of expansive in new construction, the capability of damaged concrete remains a major topic of ongoing research. The behavior of damaged model under uni-axial compressive test can be obtained from RBSM, and this gives us a possibility to connect our research the actual deterioration problems on site, for analysis their capability, prediction of structures further usage suffering from expansion.

    Furthermore, in the future, hopefully more features of concrete behavior would be introduced into the system, along with improvement in accuracy by considering more details into the local expansion mechanisms, the simulation of concrete behavior under expansion could finally lead us to undamaged analysis for concrete structures for their residual capacity, life span, and safety factor of serving under different conditions.

}

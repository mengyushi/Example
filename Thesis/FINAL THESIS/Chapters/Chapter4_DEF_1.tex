%*******10********20********30********40********50********60********70********80
\clearpage

\subsection{Residual Capabilities after Pure DEF Expansion}

Similar with ASR loading simulation, uni-axial compression test result for DEF expanded concrete model is summarised.

DEF loading simulation results of intensified center $50 \times 50 \times 50$ mm (labeled as I50), $75 \times 75 \times 75$ mm (labeled as I75) and uniformly expansion for all part (intensified center $100 \times 100 \times 100$ mm, labeled as I100), with aggregate percentage of 30\%, are presented, shown in Figure \ref{rangeagain} and Figure \ref{ahhhhh}. Alternated case for I50 with 15\% coarse aggregate is also presented.

\begin{figure}[!ht]
\centering
    %*******
    \begin{subfigure}{.33\textwidth}
      \centering
      \includegraphics[width=.8\linewidth]{Files/DEF_X/X0_3ds.png}
      \caption{Intensified $50 \times 50 \times 50$ mm Case\\ I50}
    \end{subfigure}%
    %*******
    \begin{subfigure}{.33\textwidth}
      \centering
      \includegraphics[width=.8\linewidth]{Files/DEF_X/X-5_3ds.png}
      \caption{Intensified $75 \times 75 \times 75$ mm Case \\ I75}
    \end{subfigure}%
    %*******
    \begin{subfigure}{.33\textwidth}
      \centering
      \includegraphics[width=.9\linewidth]{Files/DEF_X/X-1_3ds.png}
      \caption{Intensified $100 \times 100 \times 100$ mm Case\\ I100}
    \end{subfigure}
    %*******
  \caption{DEF intensified part range}
  \label{rangeagain}
\end{figure}

\begin{figure}[!h]
\centering
\begin{subfigure}{.5\textwidth}
  \centering
  \includegraphics[width=.4\linewidth]{Files/Aggregate/A15.png}
  \caption{15\% Coarse Aggregate}
  \label{fig:A15_model}
\end{subfigure}%
\begin{subfigure}{.5\textwidth}
  \centering
  \includegraphics[width=.4\linewidth]{Files/Aggregate/A30.png}
  \caption{30\% Coarse Aggregate}
  \label{fig:A15_model}
\end{subfigure}
\caption{Coarse Aggregate Percentage}
\label{ahhhhh}
\end{figure}

From Figure \ref{fig:A30X0FIX_LD}, Figure \ref{fig:A30X-5FIX_LD}, Figure \ref{fig:A30X-1asdasdaFIX_LD} and Figure \ref{fig:A15X0FIX_LD}, Load-Displacement relationships in fix boundary condition uni-axial compression test are shown.

%A30X0FIX

\begin{figure}[ht!]
\centering
\includegraphics[width=.8\linewidth]{Files/exp_3D/DEF/S13A30FIXX0-LOAD-DISPLACEMENT.png}
  \caption{A30 I50 Fix Load-Displacement}
  \label{fig:A30X0FIX_LD}
\end{figure}

%A30X-5FIX

\begin{figure}[ht!]
\centering
\includegraphics[width=.8\linewidth]{Files/exp_3D/DEF/S13A30FIXX-5-LOAD-DISPLACEMENT.png}
  \caption{A30 I75 Fix Load-Displacement}
  \label{fig:A30X-5FIX_LD}
\end{figure}

%A30X-1FIX

\begin{figure}[ht!]
\centering
\includegraphics[width=.8\linewidth]{Files/exp_3D/DEF/S13A30FIXX-1-LOAD-DISPLACEMENT.png}
  \caption{A15 I50 Fix Load-Displacement}
  \label{fig:A30X-1asdasdaFIX_LD}
\end{figure}

%A15X0FIX

\begin{figure}[ht!]
\centering
\includegraphics[width=.8\linewidth]{Files/exp_3D/DEF/S13A15FIXX0-LOAD-DISPLACEMENT.png}
  \caption{A15 I50 Fix Load-Displacement}
  \label{fig:A15X0FIX_LD}
\end{figure}

\clearpage

If comparing to load displacement curve obtained by Brunetaud et al., 2008\cite{Bruetaud}, shown in Figure \ref{BruetaudLoadDisp} and Figure \ref{BruetaudLoadDisp2}, it can be seen that our simulation result does show a similar trend in load displacement relationship, also a close result in maximum loading, but the slop in load-displacement cureve, which indicate elastic modulus, is relatively far from the experiental result.[Brunetaud et al., 2008]

\begin{figure}[ht!]
\centering
\includegraphics[width=.8\linewidth]{Reference/BruetaudLoadDisp2.png}
  \caption{Stress Strain relationship, 0.54\% expansion.}
  \label{BruetaudLoadDisp}
\end{figure}

\begin{figure}[ht!]
\centering
\includegraphics[width=.8\linewidth]{Reference/BruetaudLoadDisp2.png}
  \caption{Stress Strain relationship, 1.06\% expansion.}
  \label{BruetaudLoadDisp2}
\end{figure}

\clearpage

The simulation result in residual compressive strength of DEF damaged models here is plotted, as in Figure \ref{DEF_CS_summary}, which can be seen that different with ASR simulation, the decresaing trend in compressive strength is not significantly influenced by factors such as percentage of coarse aggregate, especially the intensified zone range of DEF expansion.

However, same as in ASR expansion elastic modulus simulation results, the residual elasic modulus in all DEF cases are also significantly lower than expanerimental results (Figure \ref{DEF_EM_summary}). One of the assumption is that in our simulation model, once crack is opened, it would be set as void with no abilitu to carry any force before closing the gap totally. However, this may not be true in really, for DEF, Ettringite are concentrated in some of the cracke generate. Adjustment in mechnical properties of DEF expanded concrete model may also require modification in further research.

\begin{figure}[ht!]
\centering
\includegraphics[width=.8\linewidth]{Files/CS_plot/DEFCS_all.png}
  \caption{DEF Compressive Strength Comparing With Experimental Results}
  \label{DEF_CS_summary}
\end{figure}

\begin{figure}[ht!]
\centering
\includegraphics[width=.8\linewidth]{Files/CS_plot/DEFEM_all.png}
  \caption{DEF Elastic Modulus Comparing With Experimental Results}
  \label{DEF_EM_summary}
\end{figure}


In general, the trend in residual compressive strength changing obtained from simulation is not far from the results collected from previous experimental results. With the increasing of global expansion in one-dimensional, compressive strength gradually decrease in all cases.

In some of the Load-Displacement plotting, discontinuity in Elastic Modulus is shown in the loading result, especially in A30 I75 cases and A15 I50 cases. This phenomena is also very common in damaged concrete in reality as the Elastic Modulus would change when cracks are closed horizontally.

As shown in Figure \ref{DEFA30vsA15}, if comparing 30\% coarse aggregate results(with center $50 \times 50 \times 50$ mm zone given intensified DEF expansion), which are labeled as A30 I50, and 15\% coarse aggregate results(with center $50 \times 50 \times 50$ mm zone given intensified DEF expansion), which are labeled as A15 I50, it can be seen that the residual compressive strength in 15\% coarse aggregate cases are relative lower than the 30\% coarse aggregate cases. However, if compare the residual elastic modulus of A30 I50 and A15 I50 in Figure \ref{DEF_EM_summary}, in this time, residual compressive strength in 15\% coarse aggregate cases are relative higher than the 30\% coarse aggregate cases.

\begin{figure}[ht!]
\centering
\includegraphics[width=.8\linewidth]{Files/CS_plot/DEFCS2.png}
  \caption{DEF Compressive Strength Comparing With Experimental Results}
  \label{DEFA30vsA15}
\end{figure}

However, the residual compressive strength results for same 30\% coarse aggregate models with different DEF intensified zone range are very close. A30 I100 case, as given overall uniform DEF expansion, shows slightly advantage in residual compressive strength comparing to other 2 cases with non-uniformed expansion, but this difference is only around 8\%.

\begin{figure}[ht!]
\centering
\includegraphics[width=.8\linewidth]{Files/CS_plot/DEFCS3.png}
  \caption{DEF Compressive Strength Comparing With Experimental Results}
  \label{DEF_X}
\end{figure}

\begin{figure}[ht!]
\centering
\includegraphics[width=.8\linewidth]{Files/CS_plot/DEFEM3.png}
  \caption{DEF Elastic Modulus Comparing With Experimental Results}
  \label{DEF_X_EM}
\end{figure}

While changing DEF expansion intensified zone does change the cracking pattern signigicantly, the mechanical properties such as compressive strength is not significantly changed.

However, if cross compare the trend in residual compressive strength and residual elastic modulus, shown in Figure \ref{DEF_X} and \ref{DEF_X_EM}, it can be seen that the cases with highest residual compressive strength (which is A30 I100 cases here), are having lowest residual elastic modulus. While the cases with lowest residual compressive strength (which is A30 I50 cases here), are having highest residual elastic modulus.

And in Figure \ref{compare}, the inner crack distribution changing with the increasing of global expansion is plotted. it can be seen that at same amount of global expansion, the destribution of cracks in different width zone is not the same. More larger cracks are generated in ASR expansion when same global expansion is the same. This may be one of the reason that differences exist between their residual mechanism properties like compressive strength.

\begin{figure}[ht!]
\centering
\includegraphics[width=.8\linewidth]{Files/exp_3D/Compare.png}
  \caption{Comparing crack distribution for ASR and DEF expansion}
  \label{compare}
\end{figure}

In later section, further disscussion of the relationship between cracking distribution in DEF expanded concrete model and its residual mechanical properties will also be presented.

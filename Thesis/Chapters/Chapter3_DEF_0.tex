\section{Cracking Pattern caused by Pure DEF Expansion}

\subsection{DEF Expansion Simulation of Single Aggregate Case}

In this section, similar as in ASR cases, simulation of DEF expansion on 10x10x10mm size model with only single aggregate in center of it is presented.

\begin{figure}[h!]
\centering
%*******
\begin{subfigure}{.25\textwidth}
  \centering
  \includegraphics[width=1.0\linewidth]{Files/Aggregate/A30P75.png}
\caption{3D View}
\end{subfigure}%
%*******
\begin{subfigure}{.25\textwidth}
  \centering
  \includegraphics[width=1.0\linewidth]{Files/Small_ASR/CR/DEP5-STEP(001).png}
\caption{Cross Section View}
\end{subfigure}%
%*******
\caption{Single Aggregate Case in Size 10x10x10mm}
\end{figure}


The expanse is generated at the location of interfaces between mortar and mortar elements, to introduce the expansion, as mentioned in chapter 2.

As the model is small in size, and it is a representation of the expansion in part of the whole structure in theoretically, uniform expansion is applied here.

%TODO: Single Aggregate 3D, 2D

0.0003 initial strain is introduced in each step at the interfaces between the paste and paste elements. Totally 20 steps of expansion are done.

% Small DEF CR
  \begin{figure}[ht!]
  \centering
      %*******
      \begin{subfigure}{.25\textwidth}
        \centering
        \includegraphics[width=1.0\linewidth]{Files/Small_DEF/CR/DEP5-STEP(001).png}
      \caption{Step 1}
      \end{subfigure}%
      %*******
      \begin{subfigure}{.25\textwidth}
        \centering
        \includegraphics[width=1.0\linewidth]{Files/Small_DEF/CR/DEP5-STEP(002).png}
      \caption{Step 2}
      \end{subfigure}%
      %*******
      \begin{subfigure}{.25\textwidth}
        \centering
        \includegraphics[width=1.0\linewidth]{Files/Small_DEF/CR/DEP5-STEP(003).png}
      \caption{Step 3}
      \end{subfigure}%
      %*******
      \begin{subfigure}{.25\textwidth}
        \centering
        \includegraphics[width=1.0\linewidth]{Files/Small_DEF/CR/DEP5-STEP(004).png}
      \caption{Step 4}
      \end{subfigure}

      %*******
      %*******
      \begin{subfigure}{.25\textwidth}
        \centering
        \includegraphics[width=1.0\linewidth]{Files/Small_DEF/CR/DEP5-STEP(005).png}
      \caption{Step 5}
      \end{subfigure}%
      %*******
      \begin{subfigure}{.25\textwidth}
        \centering
        \includegraphics[width=1.0\linewidth]{Files/Small_DEF/CR/DEP5-STEP(006).png}
      \caption{Step 6}
      \end{subfigure}%
      %*******
      \begin{subfigure}{.25\textwidth}
        \centering
        \includegraphics[width=1.0\linewidth]{Files/Small_DEF/CR/DEP5-STEP(007).png}
      \caption{Step 7}
      \end{subfigure}%
      %*******
      \begin{subfigure}{.25\textwidth}
        \centering
        \includegraphics[width=1.0\linewidth]{Files/Small_DEF/CR/DEP5-STEP(008).png}
      \caption{Step 4}
      \end{subfigure}

      %*******
      %*******
      \begin{subfigure}{.25\textwidth}
        \centering
        \includegraphics[width=1.0\linewidth]{Files/Small_DEF/CR/DEP5-STEP(009).png}
      \caption{Step 9}
      \end{subfigure}%
      %*******
      \begin{subfigure}{.25\textwidth}
        \centering
        \includegraphics[width=1.0\linewidth]{Files/Small_DEF/CR/DEP5-STEP(010).png}
      \caption{Step 10}
      \end{subfigure}%
      %*******
      \begin{subfigure}{.25\textwidth}
        \centering
        \includegraphics[width=1.0\linewidth]{Files/Small_DEF/CR/DEP5-STEP(011).png}
      \caption{Step 11}
      \end{subfigure}%
      %*******
      \begin{subfigure}{.25\textwidth}
        \centering
        \includegraphics[width=1.0\linewidth]{Files/Small_DEF/CR/DEP5-STEP(012).png}
      \caption{Step 12}
      \end{subfigure}

      %*******
      %*******
      \begin{subfigure}{.25\textwidth}
        \centering
        \includegraphics[width=1.0\linewidth]{Files/Small_DEF/CR/DEP5-STEP(013).png}
      \caption{Step 13}
      \end{subfigure}%
      %*******
      \begin{subfigure}{.25\textwidth}
        \centering
        \includegraphics[width=1.0\linewidth]{Files/Small_DEF/CR/DEP5-STEP(014).png}
      \caption{Step 14}
      \end{subfigure}%
      %*******
      \begin{subfigure}{.25\textwidth}
        \centering
        \includegraphics[width=1.0\linewidth]{Files/Small_DEF/CR/DEP5-STEP(015).png}
      \caption{Step 15}
      \end{subfigure}%
      %*******
      \begin{subfigure}{.25\textwidth}
        \centering
        \includegraphics[width=1.0\linewidth]{Files/Small_DEF/CR/DEP5-STEP(016).png}
      \caption{Step 16}
      \end{subfigure}

      %*******
      %*******
      \begin{subfigure}{.25\textwidth}
        \centering
        \includegraphics[width=1.0\linewidth]{Files/Small_DEF/CR/DEP5-STEP(017).png}
      \caption{Step 17}
      \end{subfigure}%
      %*******
      \begin{subfigure}{.25\textwidth}
        \centering
        \includegraphics[width=1.0\linewidth]{Files/Small_DEF/CR/DEP5-STEP(018).png}
      \caption{Step 18}
      \end{subfigure}%
      %*******
      \begin{subfigure}{.25\textwidth}
        \centering
        \includegraphics[width=1.0\linewidth]{Files/Small_DEF/CR/DEP5-STEP(019).png}
      \caption{Step 19}
      \end{subfigure}%
      %*******
      \begin{subfigure}{.25\textwidth}
        \centering
        \includegraphics[width=1.0\linewidth]{Files/Small_DEF/CR/DEP5-STEP(020).png}
      \caption{Step 20}
      \end{subfigure}

  \caption{Internal Stress in Each Step for DEF 10x10x10mm Case}
  \label{fig:DEF_Small_DEF_CR}
  \end{figure}

From the Figure \ref{fig:DEF_Small_DEF_CR}, we can see that step by step with initial stain introducing into the model, distance appears between aggregate and the surrounding element at first, meanwhile the distance between paste elements open gradually as we increase the step of expansion.

As the paste apart from aggregate in the center, there is no compressive or tension stress appearing in the aggregate. 

Not like the case of ASR, the DEF expansion here have relatively uniformed opening in all paste parts.

The total volume of this small model is increasing step by step, while until the step 20 the model expanded 0.6098\% one-dimensionally.

%TODO: Inner Stress 1-20 steps

Also, the Inner Stress condition for each step is collected, and shown in Figure %TODO:\ref{}.

Contradicted to the inner stress condition of ASR expanding case, from step 1 tension stress is generated in the aggregate, while compressive strength is generated in the paste uniformly.

This very small size simple example shows logically our method of adding initial strain to generate ASR expansion should work in the way we assumed.

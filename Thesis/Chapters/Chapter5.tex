%*******10********20********30********40********50********60********70********80

% For all chapters, use the new defined chap{} instead of chapter{}
% This will make the text at the top-left of the page be the same as the chapter

\chap{Summary and Conclusions}

In this study, numerical simulations of expansion behavior due to ASR and DEF, and their residual mechanical properties after expansion, are simulated by 3D RBSM. In this chapter, contents of the study, achievements, and tasks for the future are summarized respect of each chapter. 

In this chapter 1, detailed about research background, statement of problem and objective are explained. Some literature reviews that related to the developing.

Among many processes of concrete deterioration, Alkali-silica reaction (ASR) and Delayed Ettringite Formation (DEF) are two very common and important deterioration processes seen on concrete structures. As a result, aesthetics problems, as well as a safety concern of concrete structure suffering expansion, has become an issue. Though the surface cracking and reduction in mechanical properties in expansion deteriorated concrete structures has been confirmed on site and in the laboratory, the mechanisms for ASR and DEF and its relation with severability are still unclear. With complicated factors influencing the expanding process, the relationship between mechanical properties losses and concrete behavior during expansion become much unclear. Inner stress condition along with the inner crack generation is also difficult to obtain. Base on the long-developing term required and complicated mechanisms for this two expansion behavior, it is also difficult to analyze them quantitively with systematical results. 

In this case, a numerical simulation can be a beneficial way to understand the behavior through the study of the expanding process, as the internal stress and internal crack can be analyzed. A mesoscale analysis is proposed in this study since the cracks propagate in the 3D domain, and the 3D model with set mechanical properties of aggregate are included. Crack in mesoscale is simulated directly.

% 

In chapter 2, the method of analysis is here explained. In RBSM, a concrete member with coarse aggregate contained meshes into rigid bodies. Each rigid body is with 6 degrees of freedom and connected to all other rigid bodies by 3 springs. In order to prevent cracks propagate in the decision to choose all mesh size approximately 2x2x2 $mm^3$ is decided, to well presented the behavior including the smallest aggregates. After, the constitutive models are decided. 

In this chapter, various coarse aggregate percentage, as well as other influence factors such as ASR reactive aggregate percentage for ASR expansion and intensified expanding zone for DEF are introduced to simulate the same model to obtain a systematical result for analyzing model's expansion behavior under different conditions. 

For ASR expansion, expansion mechanism succeed in previous research done by L.EDDY in 2017 is applied, with systematical cases for different conditions. 

For DEF expansion, 

% 

In chapter 3, the details about surface cracking, inner cracking condition, as well as inner stress distribution in different expanding cases are summerized and cross compared. The constitutive models were determined to represent the material behaivior in macro-scale at each step of expansion. 

For ASR expansion, the influence of coarse aggregate percentage, the ASR reactive coarse aggregate percentage, are analysis and discussed seperately and cross compared.

While for DEF expansion, the influence of contains coarse aggregate percentage, and the influence of DEF expansion intensified zone where expansion amount given is larger, are also discussed. 


    \item Chapter 2 : Simulation model.

    In this chapter, the method in development of each numerical simulation model, related literature review of previous studies, will be described. The development of constitutive model for the expansion will be explained. The method for given expansive strain for simulating the same concrete damage will be explain step by step. The theoretical formulation related in numerical analysis of expanding behaviour are also described in this section.

    \item Chapter 3: Simulation of Cracking Pattern Of ASR and DEF Expanded Concrete.

    In this chapter, details about the surface and cross section cracking pattern results caused by ASR and/or DEF expansion are summarized, comparing between different cases and also with the experimental results.

    \item Chapter 4: Simulation of Residual Mechanical Capabilities Of ASR and DEF Expanded Concrete.

    In this chapter, details about the residual mechanical properties, including residual compressive strength and residual elastic modulus, are summarized. The relationships between residual capacity and expansion behavior due to different expansion causes are also discussed.

    \item Chapter 5: Conclusions

    In this final chapter, several remarks about the capability of ASR and/or DEF expansion simulation are emphasized. Also, some commentaries for further project or improvement are proposed.


\end{itemize}

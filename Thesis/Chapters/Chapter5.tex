%*******10********20********30********40********50********60********70********80

% For all chapters, use the new defined chap{} instead of chapter{}
% This will make the text at the top-left of the page be the same as the chapter

\chap{Summary and Conclusions}


In this study, numerical simulations of expansion behavior due to ASR and DEF, and their residual mechanical properties after expansion, are simulated by 3D RBSM. In this chapter, contents of the study, achievements, and tasks for the future are summarized respect of each chapter.

In this chapter 1, detailed about research background, statement of problem and objective are explained. Some literature reviews that related to the developing.

Among many processes of concrete deterioration, Alkali-silica reaction (ASR) and Delayed Ettringite Formation (DEF) are two very common and important deterioration processes seen on concrete structures. As a result, aesthetics problems, as well as a safety concern of concrete structure suffering expansion, has become an issue. Though the surface cracking and reduction in mechanical properties in expansion deteriorated concrete structures has been confirmed on site and in the laboratory, the mechanisms for ASR and DEF and its relation with severability are still unclear. With complicated factors influencing the expanding process, the relationship between mechanical properties losses and concrete behavior during expansion become much unclear. Inner stress condition along with the inner crack generation is also difficult to obtain. Base on the long-developing term required and complicated mechanisms for this two expansion behavior, it is also difficult to analyze them quantitively with systematical results.

In this case, a numerical simulation can be a beneficial way to understand the behavior through the study of the expanding process, as the internal stress and internal crack can be analyzed. A mesoscale analysis is proposed in this study since the cracks propagate in the 3D domain, and the 3D model with set mechanical properties of aggregate are included. Crack in mesoscale is simulated directly.

%

In chapter 2, the method of analysis is here explained. In RBSM, a concrete member with coarse aggregate contained meshes into rigid bodies. Each rigid body is with 6 degrees of freedom and connected to all other rigid bodies by 3 springs. In order to prevent cracks propagate in the decision to choose all mesh size approximately 2x2x2 $mm^3$ is decided, to well presented the behavior including the smallest aggregates. After, the constitutive models are decided.

In this chapter, various coarse aggregate percentage, as well as other influence factors such as ASR reactive aggregate percentage for ASR expansion and intensified expanding zone for DEF are introduced to simulate the same model to obtain a systematical result for analyzing model's expansion behavior under different conditions.

For ASR expansion, expansion mechanism succeeds in previous research done by L.EDDY in 2017 is applied, with systematical cases for different conditions.

For DEF expansion, non-uniformed expansion is introduced, for improving the previous mechanism. Due to the non-uniformed temperature distribution during steam curing, the inner part where experienced higher curing temperature is given more expansion here.

%

In chapter 3, the details about surface cracking, inner cracking condition, as well as inner stress distribution in different expanding cases are summarized and cross-compared. The constitutive models were determined to represent the material behavior in macro-scale at each step of expansion. The advantage of simulation in the 3D model allows us to compare the crack numerically.

For ASR expansion, the influence of coarse aggregate percentage, the ASR reactive coarse aggregate percentage, is analyzed and discussed separately and cross-compared. For all cases, characteristic map cracking pattern is presented, while the density of cracks is influenced by the amount of ASR reactive aggregate inside.

While for DEF expansion, the influence of contains coarse aggregate percentage, and the influence of DEF expansion intensified zone where expansion amount is given is larger, are also discussed. Comparing to the uninformed expanding case, all cases with center intensified expansion presented characteristic map cracking pattern. By cross-comparing, the expansion result ASR and DEF expansion, the similarities on surface cracking pattern and differences in inner cracking pattern and stress distribution is revealed, showing the difference in mechanisms does result in the different behavior in expanding process.

%

In chapter 4, the uni-axial compressive test is carried out on the expanded concrete model, due to ASR and DEF, separately. The relationships between residual capacity and expansion behavior due to different expansion causes are also discussed.

Experimental results from previous researches are collected here to analyze and compare with our simulation result.

Though the dimension and aggregate contents are not straightly constrained in our simulation, result in compressive strength comparing to the undamaged model is relatively close to some of the experimental results. However, the residual elastic modulus results are significantly lower than normal experimental results. Further adjustment may be required in material properties due to expansion in ASR and DEF, separately.

While the global expansion in one-dimensional is not straightly related to the losses in mechanical properties in our simulation, linear relationship between the number of cracks larger than 0.0005mm and the reducing in its compressive strength can be seen both in ASR expansion and DEF expansion. Which implied that while the expansion mechanism and cracking pattern can be different in ASR and DEF expansion, the residual mechanical properties, for example, the compressive strength, is strongly influenced by the existence of larger cracks.

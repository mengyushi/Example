%*******10********20********30********40********50********60********70********80

% For all chapters, use the newdefined chap{} instead of chapter{}
% This will make the text at the top-left of the page be the same as the chapter

\chap{Introduction of DEF and ASR Expansion}

Tables are good \cite{ATraveler} which is shown in \ref{table: you} and 
\ref{table: whether}!
\begin{table}[ht] 
    \centering 
    \begin{tabular}{|r|l|}
        \hline
        7C0 & hexadecimal \\
        3700 & octal \\ \cline{2-2}
        11111000000 & binary \\
        \hline \hline
        1984 & decimal \\
        \hline
    \end{tabular}
    \caption[You-table]{Table for you}
    \label{table: you} % is used to refer this table in the text 
\end{table} 

\begin{table}[ht] 
    \centering 
    \begin{tabular}{ | l | l | l | p{5cm} |}
        \hline
        Day & Min Temp & Max Temp & Summary \\ \hline
        Monday & 11C & 22C & A clear day with lots of sunshine. \\ \hline
        Tuesday & 9C & 19C & Cloudy with rain, \\ \hline
        Wednesday & 10C & 21C & Rain will still linger for the morning.\\
        \hline
    \end{tabular}
    \caption[Weather table]{Whether table for tomorrow}
    \label{table: whether} 
\end{table} 

\section{General}

\begin{figure}
    \centering
    \includegraphics[scale=1, angle=0]{Files/panther.jpg}
    \caption[The panther]{A panther is always watching}
    \label{fig: jordan}
\end{figure}

Duis nec quam quam, sed euismod justo. Pellentesque eu tellus vitae ante tempus malesuada. Nunc accumsan, quam in congue consequat, lectus lectus dapibus erat, id aliquet urna neque at massa. Nulla facilisi. Morbi ullamcorper eleifend posuere. Donec libero leo, faucibus nec bibendum at, mattis et urna. Proin consectetur, nunc ut imperdiet lobortis, magna neque tincidunt lectus, id iaculis nisi justo id nibh. Pellentesque vel sem in erat vulputate faucibus molestie ut lorem.


\begin{figure}
        \centering
        \begin{subfigure}[b]{0.4\textwidth}
                \centering        %   l   b   r   t
                \includegraphics[trim=0.3cm 0cm 0cm 0cm, clip=true, 
                                width=\linewidth]
                                {Files/panda.jpg}
                \caption{Panda}
                \label{subfig: ff}
        \end{subfigure}
        ~ % for a little horisontal distance
        \raisebox{3cm}[\height][\depth]{$\Rightarrow$}
        \hspace{0.2mm} % for a little horisontal distance
        \begin{subfigure}[b]{0.4\textwidth}
                \centering
                \includegraphics[trim=0cm 0cm 0cm 0cm, clip=true, 
                                width=\linewidth]
                                {Files/polarbear.jpg}
                \caption{Polar bear}
                \label{subfig: sf}
        \end{subfigure}
        \caption[The panda-polar bear relationship ]
                {It is not widely known that the panda becomes a polar bear 
                when dressing up in the winter camouflage suite \cite{AnExpert}}
        \label{fig: pentagram}
\end{figure}

\section{DEF}

\section{ASR}

\section{Combination}

\section{Conclusion}



